Writing software for embedded systems poses many challenges that don't exist on larger systems such as desktop computers, severs and even mobile devices.  Severely reduced cpu and memory resources, as well as minimal hardware interface abstractions, requires software to be extremely efficient and correct.  The overall objective of this project is to apply concepts in embedded systems design, as well as first principles of electronics, to construct a completely autonomous hovercraft that can navigate corners and avoid other obstacles.  A seemingly simple task such as this quickly becomes complex with the addition of hardware interface, self-monitoring and autonomous intelligence software.  

The complete project is segmented into three cumulative milestones.  Milestone 1 deals with the interfacing of the various electronic components to an Atmel AT90 microcontroller.  Milestone 2 consists of the design and construction of the hovercraft, the addition of the electronic components to the vehicle, and an analysis of various physical parameters of the hovercraft system.  Finally, milestone 3 realizes the full objective of the project: to supply the hovercraft with intelligence such that it may navigate a given course without coming in contact with any obstacles along the way.  

The graduate level criteria adds yet another objective: the communication and coordination between two hovercrafts.  Accordingly, the design, construction and testing of a second hovercraft is required, as well is supplemental communications software to support inter-vehicle synchronization.

This document describes, in detail, each of the three milestones in terms of their background theory, implementation methodologies, evaluation strategies and formalized results.  
