This project, the construction two autonomous and coordinated hovercrafts, has served as a learning experience for many problem domains including embedded software, electrical engineering, control theory, as well as many others.  This team's primary interests were those revolving around software problems - concerns such as inter-hovercraft communication, onboard software debugging, control synchronization, code efficiency and autonomous intelligence (for example the implementation of a PID controller).  Unfortunately, the team's tendencies and strengths in the software domain, and lack of knowledge in the hardware domain, was likely a main factor in the less-than-perfect project implementation.  

Arguably, the largest problem that impeded the progress of the project was the unwanted movement of each vehicle.  Factors such as rotational torque, uneven airflow from the base and skirt, nonuniform distribution of weight across the body, and unbalanced thrust from each of the propellers all contributed to unwanted movement to a hovercraft that should have been stationary.  Without this most basic level of control over the physical system, implementing a truly effective PID controller and communication among master and slave hovercrafts was next to impossible.  