

An Real Time Operating System (RTOS), is an operating system must be able to
deal with a time and event sensitive activities. The RTOS is intended intended
to be \textit{predictable} and \textit{determinante}\cite{RTOSMantis}. These
systems are also designed to limit the amount of over head that is required to
context switch between tasks. This is done my making many critical decisions
prior to runtime.



\subsection{Tasks}

There are three different types of tasks in the RTOS; system tasks, periodic tasks and round robin tasks. These tasks are ordered in decreasing priority levels. \\

\noindent{\bf System Tasks}

System tasks are the highest priority task. If a System task will preempt any other task.\\

\noindent{\bf Periodic Tasks}

Periodic tasks are the next level -- mid priority -- tasks. These tasks are scheduled to run periodically based on time (clock cycles). 
Periodic tasks are declared at the beginning of the program. A PPP array is also declared. This array allots time to each task and also declares an order for the periodic tasks. A periodic task is not allowed to take more time than has been allotted. If this occurs, it is called a timing violation, and the system halts.  \\

\noindent{\bf Round Robin Tasks}

The final level -- low priority -- are the round robin tasks. Round robin tasks are executed in the \textit{idle} time between periodic tasks.The kernel maintains a FIFO queue containing all of the round robin tasks. On each clock cycle, if the system is not attending to either a System or Periodic task then the next RR tasks is removed from the queue and is run for exactly one tick after which it is returned to the end of the FIFO queue. On the next cycle, if the higher priority tasks have not regained control, then the proceeding RR task is removed from the queue.  



\subsection{Events and Timeouts}

System and round robin tasks are capable of waiting for an event a timeout, or
either. The reason that only system and round robin tasks are allowed to wain on events, is that if a periodic task waits longer than its time allowence a timing violation would occur. \\

\noindent{\bf Events}\\
Events are initiated at the beginning of the program. Tasks that wait for an event are placed on a queue. The first item in the queue will coninue when another task \textit{signals} that event. All of the elements will be released from the queue if another task \textit{broadcasts} that event. \\

\noindent{\bf Timeouts}\\
The kernel maintains a variable \textit{now} indicating how many ticks have occured since program instantiation. When a task waits a certain amount of time, it will return normal operation after n ticks have passed.





The RTOS used for this project was created by Scott Craig and Justin Tanner \cite{RTOSSJ}. This RTOS is written in a mixture of C and assembly and is primarily written to support periodic tasks.   




\noindent\textbf{Sonar}

The integration of the sonar and the RTOS is based on the report by Will
Logan, Cambria Hanson and Jason Kereluk \cite{autoB}.




-Overall Design of our RTOS
-GANT charts
-Timming Diagrams
-Code
